\documentclass[11pt]{article} %

\usepackage[american]{babel} %for language
\usepackage[utf8]{inputenc} %utf8x to allow input of non-English characters
\usepackage[letterpaper]{geometry} %set the size of the page and margins
\usepackage{natbib} % bibliography management with bibtex
\bibliographystyle{agu} % natbib style

% some more useful packages to load
\usepackage{amsmath, % standard math library
graphicx, % input figures if needed
mathpazo, % changed fonts
setspace, % custom spacing
siunitx, % follow SI unit conventions
acronym, % automatically handle acronyms in the text
hyperref, % clickable cross-references and URLs
cleveref % automatic cross-references
} 


% Custom Macros
\newcommand{\fn}[1]{\texttt{#1}} % customize appearance of filename
\newcommand{\code}[1]{\texttt{#1}} % customize appearance of code snippets
\newcommand{\mtlb}{\texttt{Matlab\ }}
\newcommand{\usee}[1]{\textit{#1}} % usee = what you see on the website

% Custom macros relating to the paper
\newcommand{\thepaper}{Analytic Modeling of Rainwater Harvesting in the Brazilian Semi-Arid Northeast}
\newcommand{\github}{\url{https://github.com/jdossgollin/cisternas}}

% Document Information
\title{Model Replication Guide: \\ \thepaper}
\author{James Doss-Gollin \and Francisco de Assis de Souza Filho \and Francisco Osny En\'{e}as da Silva}


%-----------------------------------------
\begin{document}
\maketitle %titlepage

\begin{abstract} % summary of this document
This is a guide to the model used in the paper \thepaper \ (henceforth \cite{dossgollin2015}).
This document will explain the steps needed to replicate the findings of \cite{dossgollin2015}, including accessing and downloading the data, cleaning and formatting the data, and running the model under default assumptions or sensitivity analysis.
This document is intended to accompany the paper \cite{dossgollin2015} , which explains the choices and assumptions that drive this model.
\end{abstract}


\clearpage
\tableofcontents
\clearpage % start on a new page


\section{Quick Start}

There are some steps you must perform in order to replicate this model.
Each step is described in greater detail below.
\begin{enumerate}
\item Ensure that you have all appropriate software. See \cref{requirements} for help.
\item Register with the \ac{INMET} and download the data. We apologize that we are not able to provide you with our raw data, due to the data use policy of the \ac{INMET} which owns the data. Downloading their data is straightforward, except that the web site is in Portuguese. For step-by-step instructions please see \cref{dload}.
\item Run the model under normal conditions (\cref{defparam}) and/or run a sensitivity analysis (\cref{modparam}).
\end{enumerate}



\section{System Requirements} \label{requirements}

Access to \mtlb proprietary software is required.
This model was created and has been tested on \mtlb 2014a and 2015a, and its compatibility with other versions is not guaranteed.
Some toolkits beyond the core may be required, and specifically the Mapping Toolkit is required to draw figures (this feature is not required in order to run the program).
We request that anyone attempting to run this model on other versions of Matlab advise us of the compatibility so that we can update this document.
The model should function independently and should work on most standard operating systems: it has been tested successfully on Linux Mint 17, Microsoft Windows 8, and OS X Yosemite.


\section{Downloading the Raw Data} \label{dload}

\subsection{Data Use Policy}
We understand that the most convenient and efficient way for you to use this model is for us to include our data along with our code.
Unfortunately, doing so would violate the data use policy of the \ac{INMET}, which owns the data.
They require that each person who uses the data register with their online site.
Therefore, all we can do is provide the best possible instructions for accessing their data.

\subsection{Registering with the \acs{INMET}} \label{register}
In order to download the data from the \ac{INMET} you need to create an account with them.
Because their web page is written only in Portuguese, this section will walk you through that process.

\begin{enumerate}
\item Go to \url{http://www.inmet.gov.br/projetos/rede/pesquisa/}. If, for some reason, that web link fails then try searching for \code{BDMEP - Banco de Dados Meteorol\'{o}gicos para Ensino e Pesquisa INMET}.
\item Click the large green button that says \usee{Novo Cadastro}.
\item Fill out the resulting form as follows:

\begin{description}
\item[Usu\'{a}rio] enter your e-mail	
\item[Redigite (e-mail)] re-enter your e-mail for verification
\item[Nome] Your first name
\item[Sobrenome] Your last name
\item[Tipo (N,E e G)] leave as N (do not modify)
\item[Institui\c{c}\~{a}o] your institution (university, government agency, etc.) If none apply, write \code{N/a}
\item[Cargo/Fun\c{c}\~{a}o] your position within that institution. If none apply, write \code{researcher}
\item[Documento (RG)]
\item[CPF]
\item[Endere\c{c}o] Your work address
\item[CEP]
\item[Telefones] Your telephone number. Please enter a number beginning with +1
\item[\'{A}rea de aplica\c{c}\~{a}o dos dados] This question is requesting what you will use the data for. Please write \code{Research}. 
\end{description}

\end{enumerate}

\subsection{Downloading the Data}

There are two methods that you can use to download the data.
The first is tedious but straightforward; the other is more efficient but requires greater programming ability.

\subsubsection{Method 1: Manual Download}
\begin{enumerate}
\item Go to the \ac{INMET} \ac{BDMEP} website at \url{http://www.inmet.gov.br/projetos/rede/pesquisa/inicio.php}.
\item Enter your user name (your email address) and password
\item Under the \usee{Pesquisa -- Esta\c{c}ao Convencional} tab, click on the link that says \usee{S\'{e}rie Hist\'{o}rica -- Dados Di\'{a}rios}

\item Fill out the new dialogue box as follows:
\begin{description}
\item[Per\'{i}odo - Data inicio (dd/mm/aaaa)] Start date (dd/mm/yyyy format). \citet{dossgollin2015} uses 01/01/1950.
\item[fim] End date (dd/mm/yyyy format). \citet{dossgollin2015} uses 31/12/2013.
\item[Drop-Down Menu] the quickest option is to select the drop-down menu on the left (\usee{Regi\~{a}o)} and select \usee{Nordeste}. However, this may omit one or more of the stations used in \cite{dossgollin2015}. For full replication, you must select Cear\'{a} from the drop-down menu on the right. The following steps will then need to be repeated for the states of Rio Grande do Norte, Paraiba, Pernambuco, Alagoas, Sergipe, Bahia, Minas Gerais, Piau\'{i}, and Maranh\~{a}o.
\item[checkbox] Click the checkbox corresponding to \usee{precipita\c{c}\~{a}o (mm)}.
\item[submit] Click the button at the bottom that says \usee{pesquisa}. Confirm if you are asked to do so.
\end{description}

\item A list of stations will appear on the right. For each one, you must:
\begin{enumerate}
\item Click on the station
\item Open the resulting tab
\item Save the data to a text file with the name \code{\#\#\#\#\#.txt}, where \code{\#\#\#\#\#} is the station code (you will see \usee{(OMM: \#\#\#\#\#)}). Please note that a typo in this file name should not affect the code in any way.
\end{enumerate}
\item When you have downloaded all files, move all the text files into the \fn{data/INMET} folder. 
\end{enumerate}

\subsubsection{Download Using \code{RCurl}}

The manual download process, while effective, is undeniably cumbersome.
We have written a script using the \code{R} language and the  \code{RCurl} package to automate the download process.
The script is written to substitute for downloading files individually, and writes to file the same \code{.txt} files that manual download produces.
This process runs slowly, but you do not need to remain at your computer while it runs.
We also cannot guarantee that it will work perfectly, but we are committed to resolving any bugs with this script that arise.

To use the \code{RCurl} method, you must first have the \code{R} language installed on your computer. You can also ensure that you have the \code{RCurl} package installed with the command \code{install.packages("RCurl")}. If you are not familiar with \code{R} or the instructions below, it is probably easier to download the data files manually.

To use the script to automatically download data files from the \ac{BDMEP}:

\begin{enumerate}
\item If you have not already done so, follow the steps outlined at \cref{register} and create an account with the \ac{INMET}.
\item Go to \url{github.com/jdossgollin/inmet} and select the \usee{download as .zip} option
\item Extract the zip file and open the folder
\item Enter your email and \ac{BDMEP} password into the corresponding spaces of the \code{R} script.
\item If you wish, modify the start and/or end date in the script
\item Run the code as source
\item Move the data files generated to the \fn{/data/INMET} directory of the \mtlb program described in this document
\item For more questions about the \code{R} script to download these files, please use the features of the \fn{Github} web site.
\end{enumerate}


\section{Running the Model}

Before you can run the model, you need to download the raw data from the \ac{INMET} and move the raw text files to the \fn{data/INMET} folder.
If you have not already done that, see \cref{dload}.
Otherwise, you are ready to run the model: it is designed to work as soon as it is downloaded.

\subsection{Running with Default Parameters} \label{defparam}

All you need to do to run the model is to run the script \fn{main.m}.
It is  important not to modify the directory structure or move files out of the directory, because they call each other using relative paths.
This script prints results to screen.
By default, map generation is turned off; to change this option, change the \code{const.opt.figure toggle} option in \fn{declarations.m} to \code{1} (see \cref{modparam} for details on modifying parameters).
You should see results printed to screen giving average calculated reliabilities for different roof sizes.

\subsection{Modifying Parameters} \label{modparam}

All parameters should be modified exclusively in the \fn{declarations.m} file.
Parameters and options are designed to be passed through the struct called \code{const} and are not hard-coded into individual functions. 
\fn{declarations.m} describes each parameter and its role in the model.
If modifying {declarations.m} it is recommended to leave the default parameters commented, rather than deleting them, so that it is easier to return to the default configuration.
It is also important to note that the \fn{getData.m} script looks for two files: \fn{data.mat} and \fn{data\_unfiltered.mat}. 
If \fn{data.mat exists}, then \fn{getData.m} will simply load \fn{data.mat} and will not attempt to filter the data.
If \fn{data.mat} does not exist but \fn{data\_unfiltered.mat} does exist, it will load and filter this data but will not attempt to re-read files added to the \fn{INMET} folder.
For this reason, after changing the criteria by which the data should be filtered it is imperative to delete \fn{data.mat} and after altering the raw data from the \ac{INMET} it is imperative to delete both \fn{data.mat} and \fn{data\_unfiltered.mat}.


\subsection{Sensitivity Analyses}

To carry out sensitivity analyses, use the \fn{mainForSensitivityAnalysis.m} script. 
The script is divided into sections, each of which lays out sensitivity analyses for a different model parameter. 
To run all analyses, run the script; to run analysis for a single variable, position the cursor within that section and click the \code{Run Section} command.
To modify the values of one or more of the parameters manually, change it in this section (not in \fn{declarations.m}). 
That is because this file calls \fn{declarations.m} and over-rides the parameter for which the sensitivity analysis is being carried out.
The \fn{mainForSensitivityAnalysis.m} script displays results to the screen and creates a new folder \fn{Results} in which results for each variable are saved as \fn{.csv} files.
Note that only numbers -- no header rows or roof sizes -- are printed to the \code{.csv} files.

\section{Description of Files}
The folder containing this model is organized as follows.
There are three folders, called \fn{data} , \fn{doc} , and \fn{code}.
The contents of each are briefly summarized in this section.

\subsection{The \fn{code} Folder}
The following \mtlb files are included in the \fn{code} directory. Documentation of each can be accessed by typing \code{help fname}, where \code{fname} is the name of the file. A brief summary of each file is provided here:
\begin{description}
\item[\fn{main.m}] calls all functions required to run the model. To run the model, just run this file.
\item[\fn{mainForSensitivityAnalysis.m}] calls all functions required to run sensitivity analysis.
\item[\fn{declarations.m}] defines model options and parameters; saves them to a single struct which is passed as an argument to many other functions.
\item[\fn{getData.m}] determines what data has already been loaded and calls functions to read raw data from text file and filter data if filtered data does not already exist.
\item[\fn{readINMET.m}] function to read data files in \fn{.txt} format downloaded from the \ac{INMET} website. This script reads post number, coordinates, and first word of post name; saves precipitation data as \fn{.mat} file.
\item[\fn{makePrecipMatrix.m}] reads precipitation .mat files generated by \fn{readINMET.m} and creates a single matrix for all specified posts.
\item[\fn{filterData.m}] filters data by location and quality.
\item[\fn{loadRoofData.m}] reads roof data from one of the three files specified in \cref{dfolder}.
\item[\fn{cConsumption.m}] calculates consumption as a function of the consumption rule specified in declarations.m and day of the year.
\item[\fn{cBalance.m}] carries out water balance for set of meteorological posts.
\item[\fn{drawMaps.m}] draws maps and plots results of analysis.
\end{description}

\subsection{The \fn{data} Folder} \label{dfolder}
The data folder contains three child directories which contain separate data sets:
\begin{description}
\item[\fn{INMET}] is currently empty. Once you have downloaded the data from the \
\item[\fn{Roofs}] contains four text files containing roof data:
\begin{description}
\item[\fn{default.txt}] contains the default set of roofs used for the simulation as presented in the article.
\item[\fn{Milha.txt}] contains the full distribution of roof sizes published available as a supplemental file to the online version of the article.
\end{description}
\item[\fn{shapefile\_brasil}] contains a an ESRI-format shapefile, generated by the \ac{IBGE}, which is used to overlay the political divisions of Brazil onto maps.
\end{description}
It is worth noting that after running the model, several data files in \fn{.mat} format will be generated and saved directly in the data folder.

\subsection{The \fn{doc} Folder} \label{doc}
This \fn{doc} folder contains the documentation required for you to run the program.
A \code{.pdf} file is compatible on virtually all machines.
For those who are interested, we also provide the \code{.tex} code used to generate the \code{.pdf} document.
The \code{.bib} file containing references used in this document is also provided.


\section{Further Questions}

This model is hosted on \code{Github} and we request that all feedback or questions related to implementation of the model go through that site.
Please visit \github and click the \usee{Issue} tab.
This allows us to respond to your comment as rapidly as possible.
It also ensures that others have access to your question and our response.
We will prioritize issues that relate to bugs in the software or help needed over feature requests.



% Appendix --------------------------------------------

\clearpage
\appendix
\part*{Appendix}

\section{Acronyms Used}

\begin{acronym}[INMET] % optional parameter sets column width
\acro{INMET}{Instituto Nacional de Meteorologia -- National Meteorology Institution}
\acro{IBGE}{Instituto Brasileiro de Geografia e Estat\'{i}stica -- Brazilian Institute of Geography and Statistics}
\acro{BDMEP}{Banco de Dados Meteorol\'{o}gicos para Ensino e Pesquisa -- Meteorological Database for Research and Teaching}
\end{acronym}

\addcontentsline{toc}{section}{References} % add the bibliography to the table of contents
\bibliography{library} % references used


\end{document}